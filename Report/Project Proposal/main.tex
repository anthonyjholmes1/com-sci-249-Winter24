\documentclass{article}
\pdfoutput=1 
% if you need to pass options to natbib, use, e.g.:
%     \PassOptionsToPackage{numbers, compress}{natbib}
% before loading neurips_data_2023

% ready for submission
\usepackage[preprint]{neurips_data_2023}
\usepackage{multicol}

% to compile a preprint version, add the [preprint] option, e.g.:
%     \usepackage[preprint]{neurips_data_2023}
% This will indicate that the work is currently under review.

% to compile a camera-ready version, add the [final] option, e.g.:
%     \usepackage[final]{neurips_data_2023}

% to avoid loading the natbib package, add option nonatbib:
%    \usepackage[nonatbib]{neurips_data_2023}

% Submissions to the datasets and benchmarks are typically non anonymous,
% but anonymous submissions are allowed. If you feel that you must submit 
% anonymously, you can compile an anonymous version by adding the [anonymous] 
% option, e.g.:
%     \usepackage[anonymous]{neurips_data_2023}
% This will hide all author names.

\usepackage[utf8]{inputenc} % allow utf-8 input
\usepackage[T1]{fontenc}    % use 8-bit T1 fonts
\usepackage{hyperref}       % hyperlinks
\usepackage{url}            % simple URL typesetting
\usepackage{booktabs}       % professional-quality tables
\usepackage{amsfonts}       % blackboard math symbols
\usepackage{nicefrac}       % compact symbols for 1/2, etc.
\usepackage{microtype}      % microtypography
\usepackage{xcolor}         % colors
\usepackage{caption}
\usepackage{subcaption}
\usepackage{natbib}
\usepackage{lipsum} 

% wrappable floats with text 
\usepackage{wrapfig} 

% Comment Commands
\newcommand{\ba}[1]{{{\textcolor{red}{[B: #1]}}}}
\newcommand{\sj}[1]{{{\textcolor{blue}{[S: #1]}}}}
\newcommand{\yy}[1]{{{\textcolor{green}{[Yu: #1]}}}}
\newcommand{\hy}[1]{{{\textcolor{purple}{[H: #1]}}}}
\newcommand{\yx}[1]{{{\textcolor{orange}{[X: #1]}}}}
\newcommand{\animals}{\textsc{SpuCoAnimals}\xspace}
\newcommand{\mnist}{\textsc{SpuCoMNIST}\xspace}
% For theorems and such
\usepackage{amsmath}
\usepackage{amssymb}
\usepackage{mathtools}
\usepackage{amsthm}
\usepackage{bbm}
\usepackage{bm}
\usepackage{enumitem}
\usepackage{comment}
\usepackage{pifont}
\usepackage{xspace}


% if you use cleveref..
\usepackage[capitalize,noabbrev]{cleveref}

\usepackage[export]{adjustbox}
\usepackage{array}
\usepackage{tabularx}
\usepackage{booktabs}
\usepackage{graphicx}

\def\tabularxcolumn#1{m{#1}}

%%%%%%%%%%%%%%%%%%%%%%%%%%%%%%%%
% THEOREMS
%%%%%%%%%%%%%%%%%%%%%%%%%%%%%%%%
\theoremstyle{plain}
\newtheorem{theorem}{Theorem}[section]
\newtheorem{proposition}[theorem]{Proposition}
\newtheorem{lemma}[theorem]{Lemma}
\newtheorem{corollary}[theorem]{Corollary}
\theoremstyle{definition}
\newtheorem{definition}[theorem]{Definition}
\newtheorem{assumption}[theorem]{Assumption}
\theoremstyle{remark}
\newtheorem{remark}[theorem]{Remark}

% Todonotes is useful during development; simply uncomment the next line
%    and comment out the line below the next line to turn off comments
%\usepackage[disable,textsize=tiny]{todonotes}
\usepackage[textsize=tiny]{todonotes}

\newcommand{\CE}{\text{CE}}

\newcommand{\x}{\pmb{x}}
\newcommand{\X}{\pmb{X}}
\newcommand{\y}{\pmb{y}}
\newcommand{\w}{\pmb{w}}
\newcommand{\rr}{\pmb{r}}
\newcommand{\vb}{\pmb{v}}
\newcommand{\D}{\mathcal{D}}
\newcommand{\OO}{\mathcal{O}}
\newcommand{\LL}{\mathcal{L}}
\newcommand{\J}{\mathcal{J}}
\newcommand{\N}{\mathcal{N}}
\newcommand{\A}{\mathcal{A}}
\newcommand{\B}{\mathcal{B}}
\newcommand{\ab}{\pmb{a}}
\newcommand{\bb}{\pmb{b}}
\newcommand{\NTK}{\pmb{\Theta}}
\newcommand{\bet}{\pmb{\beta}}
\newcommand{\mtx}{\bm} % bold matrix
\newcommand{\vct}{\bm} % bold vector
\newcommand{\mino}{\text{mino}}
\newcommand{\maj}{\text{maj}}
%%%%%%%%%%%%%%%%%%%%
\newcommand{\inner}[2]{\left\langle#1,#2\right\rangle}
\newcommand{\innerproduct}[2]{\langle #1, #2 \rangle}
\newcommand{\norm}[1]{ \left\| #1 \right\| }
%\newcommand\uniform{\overset{\text{unif.}}{\sim}}
%%%%%%%%%%%%%%%%%%%%
\DeclareMathOperator{\Tr}{Tr}
\DeclareMathOperator{\poly}{poly}
\DeclareMathOperator*{\E}{\mathbb{E}}
\DeclareMathOperator*{\argmax}{arg\,max}
\DeclareMathOperator*{\argmin}{arg\,min}
%%%%%%%%%%%%%%%%%%%%
\newcommand{\cmark}{\ding{51}}%
\newcommand{\xmark}{\ding{55}}%

\newcommand{\spuco}{\textsc{SpuCo}\xspace}
\newcommand{\DISPEL}{\textsc{Dispel}\xspace}
\newcommand{\SPARE}{\textsc{Spare}\xspace}

\title{Analyzing Wildfire Prone Areas in California: A Multifactorial Approach}

% The \author macro works with any number of authors. There are two commands
% used to separate the names and addresses of multiple authors: \And and \AND.
%
% Using \And between authors leaves it to LaTeX to determine where to break the
% lines. Using \AND forces a line break at that point. So, if LaTeX puts 3 of 4
% authors names on the first line, and the last on the second line, try using
% \AND instead of \And before the third author name.

\author{%   
  Anthony Holmes anthonyjholmes1@g.ucla.edu\\
  Department of Computer Science, UCLA, Los Angeles, CA, 90024.\\
}


\begin{document}

\maketitle

% \input{abstract}

\section{Introduction}

California's diverse landscapes, ranging from dense forests to sprawling urban areas, have been increasingly threatened by wildfires. These devastating events not only pose a risk to human life and property but also have profound ecological impacts, making the study of their causative factors an urgent concern. The susceptibility of areas to wildfires is influenced by a complex interplay of factors, including but not limited to vegetation types, weather patterns, agricultural practices, and economic conditions. Gaining a deeper understanding of how these elements interact is key to coming up with plans to reduce the risk of wildfires and safeguard communities at risk. This project aims to analyze the factors that make certain areas in California more prone to wildfires than others. 

\section{Motivation}

My idea behind this study originates from my own experiences growing up in Ireland, a country that has year round mild climate, where extreme weather conditions are a rarity. The winter of 2009 to early 2010, however differed from this norm, Ireland was gripped by one of its most severe cold spells in history. Ireland is not equipped to handle sub zero temperatures. My family and I faced considerable hardship when the plummeting temperatures transformed roads into treacherous ice paths, we lived on top of a steep hill which effectively isolated us. The freezing conditions led to our water pipes bursting and the central heating system failing, leaving us without water and reliant on wood logs for warmth. We walked nearly 2 miles everyday to gather basic goods to survive this spell. This period of unforeseen hardship, lasting two weeks, deeply impacted me and ignited a commitment to contributing positively to global society.

This strengthened when I moved to California in 2018, where I was captivated by the United States' breathtaking natural landscapes, from its vast national parks and diverse wildlife to its picturesque coastlines. The destruction brought by the August Complex wildfire in 2020, with its toll on life and nature, was a turning point for me. It prompted the urgency of taking action to preserve our environment.

In my professional role today, I focus on identifying and rectifying inefficiencies in manufacturing processes, I have already begun to contribute to environmental sustainability. These efforts have led to reductions in carbon emissions, and minimized raw material usage. Pursuing a graduate degree represents a continuation of this journey, driven by my belief in the power of data to effect change. With the data seemingly within reach, my goal is to deepen my understanding of how to leverage this information effectively.

This project marks a significant step towards realizing my long-term ambition of dedicating my career to the environmental sector. It is an opportunity to apply the education that I have received so far at UCLA, and to embark on a path that I hope will allow me to make a meaningful and lasting impact on our world.

\section{Methods}
I would like to involve various aspects of material learned to date, while also anticipating that future lectures may be included in the final project report.

\begin{itemize}
    \item Linear Algebra
    \begin{itemize}
        \item Use of matrix operations for data manipulation, essential for handling and analyzing large datasets, including geographical and meteorological data.
    \end{itemize}
    \item Probability Models
    \begin{itemize}
        \item Utilization of probability distributions to model the likelihood of wildfires under different conditions, helping in risk assessment and understanding the impact of variables like weather patterns on wildfire occurrences.
    \end{itemize}
    \item Statistical Inference
    \begin{itemize}
        \item Confidence intervals and hypothesis testing to make informed conclusions about the data. This includes testing hypotheses regarding the impact of specific factors on wildfire risk and the statistical significance of observed relationships.
    \end{itemize}
    \item Hypothesis testing
    \begin{itemize}
        \item Formal hypothesis testing to evaluate the influence of various factors (e.g., vegetation types, climate conditions) on the susceptibility to wildfires. This approach helps in determining whether observed differences in wildfire occurrences across different areas are statistically significant.
    \end{itemize}
    \item Linear Regression
    \begin{itemize}
        \item Application of simple and multiple linear regression models to explore and quantify the relationship between the frequency or severity of wildfires and potential predictors such as temperature, humidity, land usage, and socio-economic factors.
        \item Estimation and interpretation of regression coefficients to understand the impact of each factor on wildfire risk, adjusting for the presence of other variables.
    \end{itemize}
\end{itemize}



\section{Datasets}

I plan to utilize a comprehensive dataset that integrates various dimensions relevant to wildfire risk assessment. This dataset will encompass:

Historical Wildfire Records: Data on past wildfires, including locations, extents, causes, and damages. This information is crucial for understanding patterns and trends in wildfire occurrences.

Weather and Climate Data: Detailed records of weather conditions such as temperature, humidity, rainfall, and wind speed. Climate data over longer periods will also be included to assess the impact of climate change on wildfire frequency and intensity.

Vegetation and Land Use Information: Data on vegetation types, forest density, and land use practices. This includes mapping of tree types, agricultural areas, and urban development, which are significant factors in wildfire behavior.

Socioeconomic and Demographic Data: Information on population density, economic conditions, and infrastructure within fire-prone regions. Socioeconomic factors can influence both the risk and impact of wildfires.

Remote Sensing and Satellite Imagery: Satellite data offering high-resolution imagery to monitor vegetation health, drought conditions, and changes in land cover. This allows for real-time analysis and historical comparison.

The dataset will be sourced from various reputable organizations and platforms, including:

\begin{enumerate}
    \item National Interagency Fire Center (NIFC) for historical wildfire records.
    \item NOAA National Centers for Environmental Information (NCEI) for weather and climate data.
    \item U.S. Geological Survey (USGS) for land use and vegetation mapping.
    \item U.S. Census Bureau for socioeconomic and demographic data.
    \item NASA Earth Observing System Data and Information System (EOSDIS) for satellite imagery.
\end{enumerate}





% \input{related_work}

% \input{problem_formulation}

% \input{method}

% \input{experiments}

% \input{conclusion}
% \newpage
% \input{contributions}

%\section*{References}
% \newpage
% \bibliography{references}
% \bibliographystyle{plainnat}

\end{document}
