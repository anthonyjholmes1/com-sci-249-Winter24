\section{Datasets}

I plan to utilize a comprehensive dataset that integrates various dimensions relevant to wildfire risk assessment. This dataset will encompass:

Historical Wildfire Records: Data on past wildfires, including locations, extents, causes, and damages. This information is crucial for understanding patterns and trends in wildfire occurrences.

Weather and Climate Data: Detailed records of weather conditions such as temperature, humidity, rainfall, and wind speed. Climate data over longer periods will also be included to assess the impact of climate change on wildfire frequency and intensity.

Vegetation and Land Use Information: Data on vegetation types, forest density, and land use practices. This includes mapping of tree types, agricultural areas, and urban development, which are significant factors in wildfire behavior.

Socioeconomic and Demographic Data: Information on population density, economic conditions, and infrastructure within fire-prone regions. Socioeconomic factors can influence both the risk and impact of wildfires.

Remote Sensing and Satellite Imagery: Satellite data offering high-resolution imagery to monitor vegetation health, drought conditions, and changes in land cover. This allows for real-time analysis and historical comparison.

The dataset will be sourced from various reputable organizations and platforms, including:

\begin{enumerate}
    \item National Interagency Fire Center (NIFC) for historical wildfire records.
    \item NOAA National Centers for Environmental Information (NCEI) for weather and climate data.
    \item U.S. Geological Survey (USGS) for land use and vegetation mapping.
    \item U.S. Census Bureau for socioeconomic and demographic data.
    \item NASA Earth Observing System Data and Information System (EOSDIS) for satellite imagery.
\end{enumerate}



