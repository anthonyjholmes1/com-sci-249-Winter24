\section{Methods}
I would like to involve various aspects of material learned to date, while also anticipating that future lectures may be included in the final project report.

\begin{itemize}
    \item Linear Algebra
    \begin{itemize}
        \item Use of matrix operations for data manipulation, essential for handling and analyzing large datasets, including geographical and meteorological data.
    \end{itemize}
    \item Probability Models
    \begin{itemize}
        \item Utilization of probability distributions to model the likelihood of wildfires under different conditions, helping in risk assessment and understanding the impact of variables like weather patterns on wildfire occurrences.
    \end{itemize}
    \item Statistical Inference
    \begin{itemize}
        \item Confidence intervals and hypothesis testing to make informed conclusions about the data. This includes testing hypotheses regarding the impact of specific factors on wildfire risk and the statistical significance of observed relationships.
    \end{itemize}
    \item Hypothesis testing
    \begin{itemize}
        \item Formal hypothesis testing to evaluate the influence of various factors (e.g., vegetation types, climate conditions) on the susceptibility to wildfires. This approach helps in determining whether observed differences in wildfire occurrences across different areas are statistically significant.
    \end{itemize}
    \item Linear Regression
    \begin{itemize}
        \item Application of simple and multiple linear regression models to explore and quantify the relationship between the frequency or severity of wildfires and potential predictors such as temperature, humidity, land usage, and socio-economic factors.
        \item Estimation and interpretation of regression coefficients to understand the impact of each factor on wildfire risk, adjusting for the presence of other variables.
    \end{itemize}
\end{itemize}

